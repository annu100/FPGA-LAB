\documentclass{article}
\usepackage[utf8]{inputenc}

\title{Assignment 3}
\author{ANNU,EE21RESCH01010}
\date{FPGA Assignment-3}

\begin{document}

\maketitle

\section{Introduction}

We have to perform the problem presented in Assignment-1 on arduino and verify the output.
\textbf{Draw the truth table for the inputs nd outputs given above and write POS expressions for it}
\textbf{Assignment-3 Implement above program in Arduino using assembly language.}
\section{SOLUTION}

\begin{table}[h] 
    \centering
    \begin{tabular}{|c|c|c|c|c|}
    \hline
    $A$&$B$&$C$&$D$&$X$  \\
    \hline
    0&0&0&0&0\\
    0&0&0&1&0\\
    0&0&1&0&0\\
    0&0&1&1&1\\
    0&1&0&0&0\\
    0&1&0&1&1\\
    0&1&1&0&0\\
    0&1&1&1&0\\
    1&0&0&0&0\\
    1&0&0&1&0\\
    1&0&1&0&0\\
    1&0&1&1&0\\
    1&1&0&0&1\\
    1&1&0&1&0\\
    1&1&1&0&0\\
    1&1&1&1&0\\
    \hline
    \end{tabular}
    \caption{Truth Table}
    \label{table:tt}
\end{table}



\newpage

\section{Code}

\begin{verbatim}
.include "m328Pdef.inc"

Start:
	ldi r17, 0b11000011 ; identifying input pins 10,11,12,13
	out DDRD,r17		; declaring pins as input
	ldi r17, 0b11111111 ;
	out PORTD,r17		; activating internal pullup for pins 10,11,12,13  
	in r17,PINB
	
	ldi r20,0b00000010
	rcall loopr
	
	ldi r21,0b00000001
	and r21,r17 ;s
	lsr r17
	ldi r22,0b00000001
	and r22,r17 ;r
	lsr r17
	ldi r23,0b00000001
	and r23,r17 ;q
	lsr r17
	ldi r24,0b00000001
	and r24,r17 ;p
            

	ldi r25,0b00000001
	ldi r26,0b00000001

	eor r26,r24     ;p'
	eor r25,r23     ;q'
	
            mov r27,r21
	com r27    ;s'
            mov r28,r22 
            com r28    ;r’
   
            
            mov r19,r26
            mov r20,r27
            and r19,r20    	;p's'
            mov r31,r19
            com r31          ;(p’s’)’
      
            
           mov r19,r24
           mov r20,r25 
           and r19,r20    	;pq’
           mov r30,r19    
           com r30          ;(pq’)’
     
            

            mov r19,r23
            mov r20,r22
            and r19,r20    	;qr
            mov r29,r19
            com r29          ;(qr)’
 
            mov r19,r29
            mov r20,r25
            mov r17,r28
            and r19,r20    	;p'q'
            and r17,r19     ;p’q’r’ 
            mov r23,r17
            com r23          ;(p’q’r’)’ 
            

           
            mov r19,r26
            mov r20,r28
            mov r17,r21
            and r19,r20    	;p'r'
            and r19,r17    ;p’r’s
            mov r27,r19 
            com r27          ;(p’r’s)’ 
                 
   
	and r31,r30
            and r31,r29
            and r31,r23
            and r31,r27




	rcall loopl

	ldi r16, 0b00100000 ;identifying output pins 2,3,4,5
	out DDRB,r16		;declaring pins as output
	out PORTB,r31		;writing output to pins 2,3,4,5
	

	rjmp Start

loopr:	lsr r17
	dec r20
	brne loopr
	ret

loopl:	lsl r31
	dec r20
	brne loopl
	ret





\end{verbatim}

\section{Result}

\begin{verbatim}
The assignment has been completed and truth table isverified.
\end{verbatim}
Implemented the above truth table in Arduino. The inputs A,B,C equivalent are displayed on seven segment display and its corresponding output is displayed with LED.\\
Steps: \\
1. Login into ubuntu and go to avra-1.3.0 folder\\
2. In avra-1.3.0 folder open src folder and write program in assign3.asm\\
3. One program was written to display LHS truth table and other to display RHS truth table.\\
4. Compile the program to generate the hex file.\\
5. After generating the hex file save it on laptop and load it in Arduino usng XLoader.\\
\end{document}
