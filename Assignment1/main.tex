\documentclass[journal,12pt,twocolumn]{IEEEtran}
%
\usepackage{setspace}
\usepackage{gensymb}

\singlespacing

\usepackage[cmex10]{amsmath}

\usepackage{amsthm}
\usepackage[latin1]{inputenc}
\usepackage{mathrsfs}
\usepackage{txfonts}
\usepackage{stfloats}
\usepackage{bm}
\usepackage{cite}
\usepackage{cases}
\usepackage{subfig}
\usepackage{karnaugh-map}
\usepackage{longtable}
\usepackage{multirow}

\usepackage{enumitem}
\usepackage{mathtools}
\usepackage{steinmetz}
\usepackage{tikz}
\usepackage{circuitikz}
\usepackage{verbatim}
\usepackage{tfrupee}
\usepackage[breaklinks=true]{hyperref}

\usepackage{tkz-euclide}

\usetikzlibrary{arrows, shapes.gates.logic.US, calc}
\usepackage{listings}
    \usepackage{color}                                            %%
    \usepackage{array}                                            %%
    \usepackage{longtable}                                        %%
    \usepackage{calc}                                             %%
    \usepackage{multirow}                                         %%
    \usepackage{hhline}                                           %%
    \usepackage{ifthen}                                           %%
    \usepackage{lscape}     
\usepackage{multicol}
\usepackage{chngcntr}
%\usepackage{tfrupee}
                                       %%
  %optionally (for landscape tables embedded in another document): %%
    \usepackage{lscape}     
\usepackage{multicol}
\usepackage{chngcntr}
%\usepackage{enumerate}
\usepackage{karnaugh-map}

%\usepackage{wasysym}
%\newcounter{MYtempeqncnt}
\DeclareMathOperator*{\Res}{Res}
%\renewcommand{\baselinestretch}{2}
\renewcommand\thesection{\arabic{section}}
\renewcommand\thesubsection{\thesection.\arabic{subsection}}
\renewcommand\thesubsubsection{\thesubsection.\arabic{subsubsection}}

\renewcommand\thesectiondis{\arabic{section}}
\renewcommand\thesubsectiondis{\thesectiondis.\arabic{subsection}}
\renewcommand\thesubsubsectiondis{\thesubsectiondis.\arabic{subsubsection}}

% correct bad hyphenation here
\hyphenation{op-tical net-works semi-conduc-tor}
\def\inputGnumericTable{}                                 %%

\lstset{
%language=C,
frame=single, 
breaklines=true,
columns=fullflexible
}
%\lstset{
%language=tex,
%frame=single, 
%breaklines=true
%}

\begin{document}
%


\newtheorem{theorem}{Theorem}[section]
\newtheorem{problem}{Problem}
\newtheorem{proposition}{Proposition}[section]
\newtheorem{lemma}{Lemma}[section]
\newtheorem{corollary}[theorem]{Corollary}
\newtheorem{example}{Example}[section]
\newtheorem{definition}[problem]{Definition}
%\newtheorem{thm}{Theorem}[section] 
%\newtheorem{defn}[thm]{Definition}
%\newtheorem{algorithm}{Algorithm}[section]
%\newtheorem{cor}{Corollary}
\newcommand{\BEQA}{\begin{eqnarray}}
\newcommand{\EEQA}{\end{eqnarray}}
\newcommand{\define}{\stackrel{\triangle}{=}}
\bibliographystyle{IEEEtran}
%\bibliographystyle{ieeetr}
\providecommand{\mbf}{\mathbf}
\providecommand{\pr}[1]{\ensuremath{\Pr\left(#1\right)}}
\providecommand{\qfunc}[1]{\ensuremath{Q\left(#1\right)}}
\providecommand{\sbrak}[1]{\ensuremath{{}\left[#1\right]}}
\providecommand{\lsbrak}[1]{\ensuremath{{}\left[#1\right.}}
\providecommand{\rsbrak}[1]{\ensuremath{{}\left.#1\right]}}
\providecommand{\brak}[1]{\ensuremath{\left(#1\right)}}
\providecommand{\lbrak}[1]{\ensuremath{\left(#1\right.}}
\providecommand{\rbrak}[1]{\ensuremath{\left.#1\right)}}
\providecommand{\cbrak}[1]{\ensuremath{\left\{#1\right\}}}
\providecommand{\lcbrak}[1]{\ensuremath{\left\{#1\right.}}
\providecommand{\rcbrak}[1]{\ensuremath{\left.#1\right\}}}
\theoremstyle{remark}
\newtheorem{rem}{Remark}
\newcommand{\sgn}{\mathop{\mathrm{sgn}}}
\providecommand{\abs}[1]{$\left\vert#1\right\vert$}
\providecommand{\res}[1]{\Res\displaylimits_{#1}} 
\providecommand{\norm}[1]{$\left\lVert#1\right\rVert$}
%\providecommand{\norm}[1]{\lVert#1\rVert}
\providecommand{\mtx}[1]{\mathbf{#1}}
\providecommand{\mean}[1]{E$\left[ #1 \right]$}
\providecommand{\fourier}{\overset{\mathcal{F}}{ \rightleftharpoons}}
%\providecommand{\hilbert}{\overset{\mathcal{H}}{ \rightleftharpoons}}
\providecommand{\system}{\overset{\mathcal{H}}{ \longleftrightarrow}}
	%\newcommand{\solution}[2]{\textbf{Solution:}{#1}}
\newcommand{\solution}{\noindent \textbf{Solution: }}
\newcommand{\cosec}{\,\text{cosec}\,}
\providecommand{\dec}[2]{\ensuremath{\overset{#1}{\underset{#2}{\gtrless}}}}
\newcommand{\myvec}[1]{\ensuremath{\begin{pmatrix}#1\end{pmatrix}}}
\newcommand{\mydet}[1]{\ensuremath{\begin{vmatrix}#1\end{vmatrix}}}
\makeatletter
\@addtoreset{figure}{problem}
\makeatother
\let\StandardTheFigure\thefigure
\let\vec\mathbf
%\renewcommand{\thefigure}{\theproblem.\arabic{figure}}
\renewcommand{\thefigure}{\theproblem}
%\setlist[enumerate,1]{before=\renewcommand\theequation{\theenumi.\arabic{equation}}
%\counterwithin{equation}{enumi}
%\renewcommand{\theequation}{\arabic{subsection}.\arabic{equation}}
\def\putbox#1#2#3{\makebox[0in][l]{\makebox[#1][l]{}\raisebox{\baselineskip}[0in][0in]{\raisebox{#2}[0in][0in]{#3}}}}
     \def\rightbox#1{\makebox[0in][r]{#1}}
     \def\centbox#1{\makebox[0in]{#1}}
     \def\topbox#1{\raisebox{-\baselineskip}[0in][0in]{#1}}
     \def\midbox#1{\raisebox{-0.5\baselineskip}[0in][0in]{#1}}
\vspace{3cm}
\title{
%	\logo{
EE 5811 : FPGA LAB \\ ASSIGNMENT 1
%	}
}
\author{ ANNU (EE21RESCH01010)}	
% make the title area
\maketitle
\newpage
%\tableofcontents
\bigskip
\renewcommand{\thefigure}{\theenumi}
\renewcommand{\thetable}{\theenumi}
%\renewcommand{\theequation}{\theenumi}
Download the codes from
\begin{lstlisting}
https://github.com/annu100/FPGA-LAB/tree/main/Assignment1
\end{lstlisting}
\section{\textbf{PROBLEM STATEMENT-ICSE 2017-5(a)}}
A school intends to select to select candidates for an inter-School Eassy competition as per the criteria given below:
\begin{itemize}
    \item The student has participated in an earlier competition and is very creative.  \\OR
    \item The student is very creative and has excellent awareness,but has not participated in any competition earlier.  \\OR
    \item The student has excellent general awareness and has won prize in an inter house competition.
\end{itemize}
The inputs are 
\begin{itemize}
    \item \textbf{A:} Participated in a competition earlier.
    \item \textbf{B:} is very creative.
     \item \textbf{C:} Won prize in an inter house competition.
      \item \textbf{D:} has excellent general awareness.

\end{itemize}

In all the above cases,1 indicates YES and 0 indicates NO .\\
\textbf{OUTPUT:} X [1 indicates YES and 0 indicates NO] .\\

\textbf{Draw the truth table for the inputs nd outputs given above and write POS expressions for it}

\section{SOLUTION}

\numberwithin{table}{section}
\begin{table}[h] 
    \centering
    \begin{tabular}{|c|c|c|c|c|}
    \hline
    $A$&$B$&$C$&$D$&$X$  \\
    \hline
    0&0&0&0&0\\
    0&0&0&1&0\\
    0&0&1&0&0\\
    0&0&1&1&1\\
    0&1&0&0&0\\
    0&1&0&1&1\\
    0&1&1&0&0\\
    0&1&1&1&0\\
    1&0&0&0&0\\
    1&0&0&1&0\\
    1&0&1&0&0\\
    1&0&1&1&0\\
    1&1&0&0&1\\
    1&1&0&1&0\\
    1&1&1&0&0\\
    1&1&1&1&0\\
    \hline
    \end{tabular}
    \caption{Truth Table}
    \label{table:tt}
\end{table}

From the truth table \ref{table:tt}, maxterms are:-
\begin{align}
    X(A,B,C,D)=\prod(0,1,2,4,6,7,8,9,10,11,13,14,15)
\end{align}

Using K-Map \ref{fig:kmap}, simplified POS expression is:
\begin{align}
X=(A+D)(\bar{A}+B)(\bar{B}+\bar{C})(A+B+C)(A+C+\bar{D})
\end{align}

\numberwithin{figure}{section}
\begin{figure}[h]
\centering
\begin{karnaugh-map}[4][4][1][$CD$][$$AB$$]
    \minterms{3,5,12}
    \maxterms{0,1,2,4,6,7,8,9,10,11,13,14,15}
    \implicant{8}{10}
    \implicant{7}{14}
    \implicant{13}{9}
    \implicant{0}{1}
    \implicantedge{0}{4}{2}{6}
    
    \draw[color=black, ultra thin] (0, 4) --
    node [pos=0.7, above right, anchor=south west] {$CD$} % YOU CAN CHANGE NAME OF VAR HERE, THE $X$ IS USED FOR ITALICS
    node [pos=0.7, below left, anchor=north east] {$AB$} % SAME FOR THIS
    ++(135:1);
    
\end{karnaugh-map}
\caption{Karnaugh-Map}
\label{fig:kmap}
\end{figure}

\subsection{Using Nand Logic:}

\begin{align}
X &=(A+D)(\bar{A}+B)(\bar{B}+\bar{C})(A+B+C)(A+C+\bar{D})\\
  &=(\bar{A}\bar{D})'(A\bar{B})'(BC)'(\bar{A}\bar{B}\bar{C})'(\bar{A}\bar{C}D)'\\
\end{align}

Now we can draw the logic circuit using NAND gates as below.\\\\


\begin{figure}[h!]
    \centering
\begin{circuitikz}[label distance=2mm, scale=2,
  connection/.style={draw,circle,fill=black,inner sep=1.5pt}
  ]
\node (x) at (0.5,0) {$A$};
\node (y) at (1,0) {$B$};
\node (z) at (1.5,0) {$C$};
\node (w) at (2.0,0) {$D$};
\node[nand gate US, draw, rotate=0, logic gate inputs=inni, scale=1.5] at ($(z)+(2,-1)$) (t1) {};
\node[nand gate US, draw, rotate=0, logic gate inputs=ninn, scale=1.5] at ($(z)+(2,-2)$) (t2) {};
\node[nand gate US, draw, rotate=0, logic gate inputs=nnnn, scale=1.5] at ($(z)+(2,-3)$) (t3) {};
\node[nand gate US, draw, rotate=0, logic gate inputs=iiin, scale=1.5] at ($(z)+(2,-4)$) (t4) {};
\node[nand gate US, draw, rotate=0, logic gate inputs=inin, scale=1.5] at ($(z)+(2,-5)$) (t5) {};
\node[nand gate US, draw, logic gate inputs=nnnnn, scale=1.25] at ($(t3.output) + (2, 0.5)$) (orTot) {};
\node[nand gate US, draw, logic gate inputs=nnnnn, scale=1.25] at ($(t3.output) + (4, 0.5)$) (orTot1) {};
\draw (x) -- ($(x) + (0,-5.5)$);
\draw (y) -- ($(y) + (0,-5.5)$);
\draw (z) -- ($(z) + (0,-5.5)$);
\draw (w) -- ($(w) + (0,-5.5)$);
\draw (x) |- (t1.input 1) node[connection,pos=0.5]{};
\draw (w) |- (t1.input 4) node[connection,pos=0.5]{};

\draw (x) |- (t2.input 1) node[connection,pos=0.5]{};
\draw (y) |- (t2.input 2) node[connection,pos=0.5]{};

\draw (y) |- (t3.input 2) node[connection,pos=0.5]{};
\draw (z) |- (t3.input 3) node[connection,pos=0.5]{};

\draw (x) |- (t4.input 1) node[connection,pos=0.5]{};
\draw (y) |- (t4.input 2) node[connection,pos=0.5]{};
\draw (z) |- (t4.input 3) node[connection,pos=0.5]{};

\draw (x) |- (t5.input 1) node[connection,pos=0.5]{};
\draw (z) |- (t5.input 3) node[connection,pos=0.5]{};
\draw (w) |- (t5.input 4) node[connection,pos=0.5]{};

\draw (t1.output) -- ([xshift=0.3cm]t1.output) |- (orTot.input 1);
\draw (t2.output) -- ([xshift=0.2cm]t2.output) |- (orTot.input 2);
\draw (t3.output) -- ([xshift=0.2cm]t3.output) |- (orTot.input 3);
\draw (t4.output) -- ([xshift=0.3cm]t4.output) |- (orTot.input 4);
\draw (t5.output) -- ([xshift=0.4cm]t5.output) |- (orTot.input 5);

\draw (orTot.output) -- ([xshift=0.4cm]orTot.output) |- (orTot1.input 5);
\draw (orTot.output) -- ([xshift=0.4cm]orTot.output) |- (orTot1.input 1);

\draw (orTot1.output) -- node[above]{$X$} ($(orTot1) + (1, 0)$);
\end{circuitikz}
\caption{Logic Circuit using NAND gates}
\label{ckt1}
\end{figure}


\end{document}
